% Options for packages loaded elsewhere
\PassOptionsToPackage{unicode}{hyperref}
\PassOptionsToPackage{hyphens}{url}
\PassOptionsToPackage{dvipsnames,svgnames,x11names}{xcolor}
%
\documentclass[
  letterpaper,
  DIV=11,
  numbers=noendperiod]{scrreprt}

\usepackage{amsmath,amssymb}
\usepackage{iftex}
\ifPDFTeX
  \usepackage[T1]{fontenc}
  \usepackage[utf8]{inputenc}
  \usepackage{textcomp} % provide euro and other symbols
\else % if luatex or xetex
  \usepackage{unicode-math}
  \defaultfontfeatures{Scale=MatchLowercase}
  \defaultfontfeatures[\rmfamily]{Ligatures=TeX,Scale=1}
\fi
\usepackage{lmodern}
\ifPDFTeX\else  
    % xetex/luatex font selection
\fi
% Use upquote if available, for straight quotes in verbatim environments
\IfFileExists{upquote.sty}{\usepackage{upquote}}{}
\IfFileExists{microtype.sty}{% use microtype if available
  \usepackage[]{microtype}
  \UseMicrotypeSet[protrusion]{basicmath} % disable protrusion for tt fonts
}{}
\makeatletter
\@ifundefined{KOMAClassName}{% if non-KOMA class
  \IfFileExists{parskip.sty}{%
    \usepackage{parskip}
  }{% else
    \setlength{\parindent}{0pt}
    \setlength{\parskip}{6pt plus 2pt minus 1pt}}
}{% if KOMA class
  \KOMAoptions{parskip=half}}
\makeatother
\usepackage{xcolor}
\setlength{\emergencystretch}{3em} % prevent overfull lines
\setcounter{secnumdepth}{5}
% Make \paragraph and \subparagraph free-standing
\ifx\paragraph\undefined\else
  \let\oldparagraph\paragraph
  \renewcommand{\paragraph}[1]{\oldparagraph{#1}\mbox{}}
\fi
\ifx\subparagraph\undefined\else
  \let\oldsubparagraph\subparagraph
  \renewcommand{\subparagraph}[1]{\oldsubparagraph{#1}\mbox{}}
\fi

\usepackage{color}
\usepackage{fancyvrb}
\newcommand{\VerbBar}{|}
\newcommand{\VERB}{\Verb[commandchars=\\\{\}]}
\DefineVerbatimEnvironment{Highlighting}{Verbatim}{commandchars=\\\{\}}
% Add ',fontsize=\small' for more characters per line
\usepackage{framed}
\definecolor{shadecolor}{RGB}{241,243,245}
\newenvironment{Shaded}{\begin{snugshade}}{\end{snugshade}}
\newcommand{\AlertTok}[1]{\textcolor[rgb]{0.68,0.00,0.00}{#1}}
\newcommand{\AnnotationTok}[1]{\textcolor[rgb]{0.37,0.37,0.37}{#1}}
\newcommand{\AttributeTok}[1]{\textcolor[rgb]{0.40,0.45,0.13}{#1}}
\newcommand{\BaseNTok}[1]{\textcolor[rgb]{0.68,0.00,0.00}{#1}}
\newcommand{\BuiltInTok}[1]{\textcolor[rgb]{0.00,0.23,0.31}{#1}}
\newcommand{\CharTok}[1]{\textcolor[rgb]{0.13,0.47,0.30}{#1}}
\newcommand{\CommentTok}[1]{\textcolor[rgb]{0.37,0.37,0.37}{#1}}
\newcommand{\CommentVarTok}[1]{\textcolor[rgb]{0.37,0.37,0.37}{\textit{#1}}}
\newcommand{\ConstantTok}[1]{\textcolor[rgb]{0.56,0.35,0.01}{#1}}
\newcommand{\ControlFlowTok}[1]{\textcolor[rgb]{0.00,0.23,0.31}{#1}}
\newcommand{\DataTypeTok}[1]{\textcolor[rgb]{0.68,0.00,0.00}{#1}}
\newcommand{\DecValTok}[1]{\textcolor[rgb]{0.68,0.00,0.00}{#1}}
\newcommand{\DocumentationTok}[1]{\textcolor[rgb]{0.37,0.37,0.37}{\textit{#1}}}
\newcommand{\ErrorTok}[1]{\textcolor[rgb]{0.68,0.00,0.00}{#1}}
\newcommand{\ExtensionTok}[1]{\textcolor[rgb]{0.00,0.23,0.31}{#1}}
\newcommand{\FloatTok}[1]{\textcolor[rgb]{0.68,0.00,0.00}{#1}}
\newcommand{\FunctionTok}[1]{\textcolor[rgb]{0.28,0.35,0.67}{#1}}
\newcommand{\ImportTok}[1]{\textcolor[rgb]{0.00,0.46,0.62}{#1}}
\newcommand{\InformationTok}[1]{\textcolor[rgb]{0.37,0.37,0.37}{#1}}
\newcommand{\KeywordTok}[1]{\textcolor[rgb]{0.00,0.23,0.31}{#1}}
\newcommand{\NormalTok}[1]{\textcolor[rgb]{0.00,0.23,0.31}{#1}}
\newcommand{\OperatorTok}[1]{\textcolor[rgb]{0.37,0.37,0.37}{#1}}
\newcommand{\OtherTok}[1]{\textcolor[rgb]{0.00,0.23,0.31}{#1}}
\newcommand{\PreprocessorTok}[1]{\textcolor[rgb]{0.68,0.00,0.00}{#1}}
\newcommand{\RegionMarkerTok}[1]{\textcolor[rgb]{0.00,0.23,0.31}{#1}}
\newcommand{\SpecialCharTok}[1]{\textcolor[rgb]{0.37,0.37,0.37}{#1}}
\newcommand{\SpecialStringTok}[1]{\textcolor[rgb]{0.13,0.47,0.30}{#1}}
\newcommand{\StringTok}[1]{\textcolor[rgb]{0.13,0.47,0.30}{#1}}
\newcommand{\VariableTok}[1]{\textcolor[rgb]{0.07,0.07,0.07}{#1}}
\newcommand{\VerbatimStringTok}[1]{\textcolor[rgb]{0.13,0.47,0.30}{#1}}
\newcommand{\WarningTok}[1]{\textcolor[rgb]{0.37,0.37,0.37}{\textit{#1}}}

\providecommand{\tightlist}{%
  \setlength{\itemsep}{0pt}\setlength{\parskip}{0pt}}\usepackage{longtable,booktabs,array}
\usepackage{calc} % for calculating minipage widths
% Correct order of tables after \paragraph or \subparagraph
\usepackage{etoolbox}
\makeatletter
\patchcmd\longtable{\par}{\if@noskipsec\mbox{}\fi\par}{}{}
\makeatother
% Allow footnotes in longtable head/foot
\IfFileExists{footnotehyper.sty}{\usepackage{footnotehyper}}{\usepackage{footnote}}
\makesavenoteenv{longtable}
\usepackage{graphicx}
\makeatletter
\def\maxwidth{\ifdim\Gin@nat@width>\linewidth\linewidth\else\Gin@nat@width\fi}
\def\maxheight{\ifdim\Gin@nat@height>\textheight\textheight\else\Gin@nat@height\fi}
\makeatother
% Scale images if necessary, so that they will not overflow the page
% margins by default, and it is still possible to overwrite the defaults
% using explicit options in \includegraphics[width, height, ...]{}
\setkeys{Gin}{width=\maxwidth,height=\maxheight,keepaspectratio}
% Set default figure placement to htbp
\makeatletter
\def\fps@figure{htbp}
\makeatother
% definitions for citeproc citations
\NewDocumentCommand\citeproctext{}{}
\NewDocumentCommand\citeproc{mm}{%
  \begingroup\def\citeproctext{#2}\cite{#1}\endgroup}
\makeatletter
 % allow citations to break across lines
 \let\@cite@ofmt\@firstofone
 % avoid brackets around text for \cite:
 \def\@biblabel#1{}
 \def\@cite#1#2{{#1\if@tempswa , #2\fi}}
\makeatother
\newlength{\cslhangindent}
\setlength{\cslhangindent}{1.5em}
\newlength{\csllabelwidth}
\setlength{\csllabelwidth}{3em}
\newenvironment{CSLReferences}[2] % #1 hanging-indent, #2 entry-spacing
 {\begin{list}{}{%
  \setlength{\itemindent}{0pt}
  \setlength{\leftmargin}{0pt}
  \setlength{\parsep}{0pt}
  % turn on hanging indent if param 1 is 1
  \ifodd #1
   \setlength{\leftmargin}{\cslhangindent}
   \setlength{\itemindent}{-1\cslhangindent}
  \fi
  % set entry spacing
  \setlength{\itemsep}{#2\baselineskip}}}
 {\end{list}}
\usepackage{calc}
\newcommand{\CSLBlock}[1]{\hfill\break\parbox[t]{\linewidth}{\strut\ignorespaces#1\strut}}
\newcommand{\CSLLeftMargin}[1]{\parbox[t]{\csllabelwidth}{\strut#1\strut}}
\newcommand{\CSLRightInline}[1]{\parbox[t]{\linewidth - \csllabelwidth}{\strut#1\strut}}
\newcommand{\CSLIndent}[1]{\hspace{\cslhangindent}#1}

\KOMAoption{captions}{tableheading}
\makeatletter
\@ifpackageloaded{bookmark}{}{\usepackage{bookmark}}
\makeatother
\makeatletter
\@ifpackageloaded{caption}{}{\usepackage{caption}}
\AtBeginDocument{%
\ifdefined\contentsname
  \renewcommand*\contentsname{Table of contents}
\else
  \newcommand\contentsname{Table of contents}
\fi
\ifdefined\listfigurename
  \renewcommand*\listfigurename{List of Figures}
\else
  \newcommand\listfigurename{List of Figures}
\fi
\ifdefined\listtablename
  \renewcommand*\listtablename{List of Tables}
\else
  \newcommand\listtablename{List of Tables}
\fi
\ifdefined\figurename
  \renewcommand*\figurename{Figure}
\else
  \newcommand\figurename{Figure}
\fi
\ifdefined\tablename
  \renewcommand*\tablename{Table}
\else
  \newcommand\tablename{Table}
\fi
}
\@ifpackageloaded{float}{}{\usepackage{float}}
\floatstyle{ruled}
\@ifundefined{c@chapter}{\newfloat{codelisting}{h}{lop}}{\newfloat{codelisting}{h}{lop}[chapter]}
\floatname{codelisting}{Listing}
\newcommand*\listoflistings{\listof{codelisting}{List of Listings}}
\makeatother
\makeatletter
\makeatother
\makeatletter
\@ifpackageloaded{caption}{}{\usepackage{caption}}
\@ifpackageloaded{subcaption}{}{\usepackage{subcaption}}
\makeatother
\ifLuaTeX
  \usepackage{selnolig}  % disable illegal ligatures
\fi
\usepackage{bookmark}

\IfFileExists{xurl.sty}{\usepackage{xurl}}{} % add URL line breaks if available
\urlstyle{same} % disable monospaced font for URLs
\hypersetup{
  pdftitle={Lpi},
  pdfauthor={Norah Jones},
  colorlinks=true,
  linkcolor={blue},
  filecolor={Maroon},
  citecolor={Blue},
  urlcolor={Blue},
  pdfcreator={LaTeX via pandoc}}

\title{Lpi}
\author{Norah Jones}
\date{2024-07-03}

\begin{document}
\maketitle

\renewcommand*\contentsname{Table of contents}
{
\hypersetup{linkcolor=}
\setcounter{tocdepth}{2}
\tableofcontents
}
\bookmarksetup{startatroot}

\chapter*{Preface}\label{preface}
\addcontentsline{toc}{chapter}{Preface}

\markboth{Preface}{Preface}

This is a Quarto book.

To learn more about Quarto books visit
\url{https://quarto.org/docs/books}.

\bookmarksetup{startatroot}

\chapter{Introdução}\label{introduuxe7uxe3o}

Material desenvolvido como apoio para a unidade curricular de Linguagem
de Programação I.

\bookmarksetup{startatroot}

\chapter{Lógica de programação}\label{luxf3gica-de-programauxe7uxe3o}

O quê é a lógica?

Estuda a ``correção do raciocínio'', tem em vista a ``ordem da razão''\\
(Forbellone, 2006, pg 1).

\textbf{Exemplos}

\begin{verbatim}
Todo mamífero é um animal.
Todo cavalo é um mamífero.
Portanto, Todo cavalo é um animal
\end{verbatim}

(Forbellone, 2006, pg 1)

\begin{verbatim}
Paraná um estado do Brasil.
Todos os cascavelenses são paranaenses.
Logo, todos os cascavelenses são brasileiros.
\end{verbatim}

A Lógica de programação consiste na ordem da razão e organização de
processos de raciocínio e simbolização formais na programação de
sistemas computacionais.

\bookmarksetup{startatroot}

\chapter{Algoritmo}\label{algoritmo}

Um \textbf{algoritmo} é uma sequência de passos organizada, de maneira
que a \textbf{execução} dos passos possibilita atingir um objetivo.

Junto à idéia de algoritmo vem a noção de \textbf{ordem}, na execução
dos passos.

\textbf{Exemplo} - Como faríamos para trocar uma lâmpada?

Resposta

\begin{Shaded}
\begin{Highlighting}[]
\ExtensionTok{pegar}\NormalTok{ uma escada}
\ExtensionTok{colocar}\NormalTok{ a escada abaixo da lâmpada}
\ExtensionTok{pegar}\NormalTok{ uma lâmpada nova}
\ExtensionTok{subir}\NormalTok{ na escada}
\ExtensionTok{retirar}\NormalTok{ a lâmpada velha}
\ExtensionTok{colocar}\NormalTok{ a lâmpada nova}
\end{Highlighting}
\end{Shaded}

\textbf{Exercício}

\begin{itemize}
\tightlist
\item
  Abaixo há um algoritmo para trocar uma lâmpada.

  \begin{itemize}
  \tightlist
  \item
    Ele resolve o problema?\\
  \item
    O que há de errado?
  \end{itemize}
\end{itemize}

\begin{verbatim}
colocar a escada abaixo da lâmpada
pegar uma escada
retirar a lâmpada velha
subir na escada
colocar a lâmpada nova
pegar uma lâmpada nova
\end{verbatim}

Resposta

A ordem está incorreta. Desta forma, o algoritmo não resolve o problema.

\textbf{Exercícios}

\begin{itemize}
\tightlist
\item
  Desenvolva um algoritmo para fazer gelo.
\item
  Desenvolva um algoritmo para fazer café.
\item
  Desenvolva um algoritmo para trocar o pneu de um carro.
\end{itemize}

\bookmarksetup{startatroot}

\chapter{Linguagem de programação}\label{linguagem-de-programauxe7uxe3o}

Uma \textbf{linguagem de programação} é um conjunto de funções
específicas e bem definidas, com as quais é possível desenvolver um
\textbf{algoritmo}.

Chamamos de \textbf{código} a um algoritmo desenvolvido em uma linguagem
de programação.

\textbf{Atividade}

\begin{itemize}
\tightlist
\item
  Um homem precisa atravessar um rio com um barco que possui capacidade
  de transportar apenas ele mesmo e mais uma de suas três cargas, que
  são: um lobo, um bode e uma caixa de alfafa. Indique as ações
  necessárias para que o homem consiga atravessar o rio sem perder suas
  cargas. Algumas regras devem ser sempre respeitadas: o lobo não pode
  ficar sozinho com o bode e o bode não pode ficar sozinho com a alfafa.

  \begin{itemize}
  \tightlist
  \item
    Desenvolva um algoritmo para a solução desse problema.
  \end{itemize}
\end{itemize}

\textbf{Atividade}

Um homem precisa atravessar um rio com um barco que possui capacidade de
transportar apenas ele mesmo e mais uma de suas três cargas, que são: um
lobo, um bode e uma caixa de alfafa. Indique as ações necessárias para
que o homem consiga atravessar o rio sem perder suas cargas. Algumas
regras devem ser sempre respeitadas: o lobo não pode ficar sozinho com o
bode, o bode não pode ficar sozinho com a alfafa.

\begin{itemize}
\tightlist
\item
  Utilize apenas a função \texttt{atravessar(p)}

  \begin{itemize}
  \tightlist
  \item
    p, é o passageiro ou carga do barco, que podem ser \texttt{lobo},
    \texttt{bode}, \texttt{alfafa} ou \texttt{ninguém};
  \item
    a função \texttt{atravessar} vai de um lado a outro, partindo do
    lado em que o barco está.
  \end{itemize}
\end{itemize}

\textbf{Atividade}

Elabore um algoritmo que mova três discos de uma Torre de Hanói, que
consiste em três hastes (A, B e C), uma das quais serve de suporte para
três discos de tamanhos diferentes (1, 2 e 3), os menores sobre os
maiores. É possível mover um disco de cada vez para qualquer haste,
porém nunca deve ser colocado um disco maior sobre um menor. O objetivo
é transferir os três discos para outra haste. No exercício, considere
que os discos a serem movimentados estão inicialmente na haste A e devem
ser movidos para a haste C.

\begin{itemize}
\tightlist
\item
  Utilize a função \texttt{movimentar(de,\ para)}, que movimenta o disco
  que está no topo da haste \texttt{de} e o coloca no topo da haste
  \texttt{para}. Apenas o disco que está no topo da haste de origem é
  movimentado.
\end{itemize}

\includegraphics{index_files/mediabag/800px-Linalg_towers_.png}\\
Imagem:
\href{https://commons.wikimedia.org/wiki/File:Linalg_towers_of_hanoi_1.png}{Wikimedia
Commons}

\bookmarksetup{startatroot}

\chapter{Funções de entrada e
saída}\label{funuxe7uxf5es-de-entrada-e-sauxedda}

import Tabs from `(\textbf{theme/Tabs?})'; import TabItem from
`(\textbf{theme/TabItem?})';

As funções de entrada e saída realizam a conexão do algoritmo com o
ambiente externo, recebendo e enviando dados.

\section{Saída de dados}\label{sauxedda-de-dados}

A saída de dados que utilizaremos aqui é a tela. Para mostrar uma
mensagem na tela, utilizaremos a função abaixo, acompanhada da mensagem
que desejamos apresentar na tela.

\begin{Shaded}
\begin{Highlighting}[]
\NormalTok{escreva}\OperatorTok{();}
\end{Highlighting}
\end{Shaded}

\begin{Shaded}
\begin{Highlighting}[]
\NormalTok{System}\OperatorTok{.}\AttributeTok{out}\OperatorTok{.}\FunctionTok{println}\NormalTok{()}\OperatorTok{;}
\end{Highlighting}
\end{Shaded}

\begin{Shaded}
\begin{Highlighting}[]
\BuiltInTok{print}\NormalTok{()}
\end{Highlighting}
\end{Shaded}

\begin{Shaded}
\begin{Highlighting}[]
\NormalTok{printf}\OperatorTok{();}
\end{Highlighting}
\end{Shaded}

:::caution Atenção Para que a função de saída \texttt{printf()} possa
ser utilizada é necessário importar a biblioteca de entrada e saída
padrão.

Para tal, basta incluir no início do código a linha

\begin{Shaded}
\begin{Highlighting}[]
\PreprocessorTok{\#include }\ImportTok{\textless{}stdio.h\textgreater{}}
\end{Highlighting}
\end{Shaded}

:::

\textbf{Exemplo}\\
- Para apresentar na tela a mensagem \texttt{Olá\ Mundo}:

\begin{Shaded}
\begin{Highlighting}[]
\NormalTok{escreva}\OperatorTok{(}\StringTok{"Olá Mundo"}\OperatorTok{);}
\end{Highlighting}
\end{Shaded}

\begin{Shaded}
\begin{Highlighting}[]
\NormalTok{System}\OperatorTok{.}\AttributeTok{out}\OperatorTok{.}\FunctionTok{println}\NormalTok{(}\StringTok{"Olá Mundo"}\NormalTok{)}\OperatorTok{;}
\end{Highlighting}
\end{Shaded}

\begin{Shaded}
\begin{Highlighting}[]
\BuiltInTok{print}\NormalTok{(}\StringTok{"Olá Mundo"}\NormalTok{)}
\end{Highlighting}
\end{Shaded}

\begin{Shaded}
\begin{Highlighting}[]
\NormalTok{printf}\OperatorTok{(}\StringTok{"Olá Mundo"}\OperatorTok{);}
\end{Highlighting}
\end{Shaded}

Saída na tela:

\begin{verbatim}
Olá Mundo
\end{verbatim}

:::caution Atenção

Perceba que aqui estamos observando apenas a função de saída. Para que
ela possa funcionar de maneira correta no algoritmo, é necessário que
ela esteja dentro de uma \textbf{estrutura básica}.

\begin{Shaded}
\begin{Highlighting}[]
\NormalTok{início}
\NormalTok{  módulo Principal}
\NormalTok{    escreva}\OperatorTok{(}\StringTok{"Olá Mundo"}\OperatorTok{);}
\NormalTok{  fimmódulo}\OperatorTok{;}
\NormalTok{fim}\OperatorTok{.}
\end{Highlighting}
\end{Shaded}

\begin{Shaded}
\begin{Highlighting}[]
\KeywordTok{public} \KeywordTok{class}\NormalTok{ Main\{}
  \KeywordTok{public} \KeywordTok{static} \KeywordTok{void} \FunctionTok{main}\NormalTok{(}\BuiltInTok{String}\NormalTok{ args)\{}
\NormalTok{    System}\OperatorTok{.}\AttributeTok{out}\OperatorTok{.}\FunctionTok{println}\NormalTok{(}\StringTok{"Olá Mundo"}\NormalTok{)}\OperatorTok{;}
\NormalTok{  \}}
\NormalTok{\}}
\end{Highlighting}
\end{Shaded}

\begin{Shaded}
\begin{Highlighting}[]
\ControlFlowTok{if} \VariableTok{\_\_name\_\_} \OperatorTok{==} \StringTok{"main"}\NormalTok{:}
  \BuiltInTok{print}\NormalTok{(}\StringTok{"Olá Mundo"}\NormalTok{)}
\end{Highlighting}
\end{Shaded}

\begin{Shaded}
\begin{Highlighting}[]
\PreprocessorTok{\#include }\ImportTok{\textless{}stdio.h\textgreater{}}
\DataTypeTok{int}\NormalTok{ main}\OperatorTok{(}\DataTypeTok{void}\OperatorTok{)\{}
\NormalTok{  printf}\OperatorTok{(}\StringTok{"Olá Mundo}\SpecialCharTok{\textbackslash{}n}\StringTok{"}\OperatorTok{);} 
  \ControlFlowTok{return} \DecValTok{0}\OperatorTok{;}
\OperatorTok{\}}
\end{Highlighting}
\end{Shaded}

Saída na tela:

\begin{verbatim}
Olá Mundo
\end{verbatim}

:::

\textbf{Exemplo}

\begin{itemize}
\tightlist
\item
  Para apresentar a mensagem \texttt{Bom\ dia,\ José}:
\end{itemize}

\begin{Shaded}
\begin{Highlighting}[]
\NormalTok{escreva}\OperatorTok{(}\StringTok{"Bom dia, José"}\OperatorTok{);}
\end{Highlighting}
\end{Shaded}

\begin{Shaded}
\begin{Highlighting}[]
\NormalTok{System}\OperatorTok{.}\AttributeTok{out}\OperatorTok{.}\FunctionTok{println}\NormalTok{(}\StringTok{"Bom dia, José"}\NormalTok{)}\OperatorTok{;}
\end{Highlighting}
\end{Shaded}

\begin{Shaded}
\begin{Highlighting}[]
\BuiltInTok{print}\NormalTok{(}\StringTok{"Bom dia, José"}\NormalTok{)}
\end{Highlighting}
\end{Shaded}

\begin{Shaded}
\begin{Highlighting}[]
\NormalTok{printf}\OperatorTok{(}\StringTok{"Bom dia, José}\SpecialCharTok{\textbackslash{}n}\StringTok{"}\OperatorTok{);}
\end{Highlighting}
\end{Shaded}

Saída na tela:

\begin{verbatim}
Bom dia, José
\end{verbatim}

\textbf{Exercício} - Desenvolva um algoritmo que apresenta seu nome na
saída padrão.

\begin{itemize}
\tightlist
\item
  Para apresentar um valor específico:
\end{itemize}

\begin{Shaded}
\begin{Highlighting}[]
\NormalTok{escreva}\OperatorTok{(}\DecValTok{18}\OperatorTok{);}
\end{Highlighting}
\end{Shaded}

\begin{Shaded}
\begin{Highlighting}[]
\NormalTok{System}\OperatorTok{.}\AttributeTok{out}\OperatorTok{.}\FunctionTok{println}\NormalTok{(}\DecValTok{18}\NormalTok{)}\OperatorTok{;}
\end{Highlighting}
\end{Shaded}

\begin{Shaded}
\begin{Highlighting}[]
\BuiltInTok{print}\NormalTok{(}\DecValTok{18}\NormalTok{)}
\end{Highlighting}
\end{Shaded}

\begin{Shaded}
\begin{Highlighting}[]
\NormalTok{printf}\OperatorTok{(}\StringTok{"}\SpecialCharTok{\%d\textbackslash{}n}\StringTok{"}\OperatorTok{,} \DecValTok{18}\OperatorTok{);}
\end{Highlighting}
\end{Shaded}

Saída na tela:

\begin{verbatim}
18
\end{verbatim}

\begin{itemize}
\tightlist
\item
  Pode-se também apresentar diversos resultados de uma única vez:
\end{itemize}

\begin{Shaded}
\begin{Highlighting}[]
\NormalTok{escreva}\OperatorTok{(}\StringTok{"A idade mínima deve ser de "}\OperatorTok{,} \DecValTok{18}\OperatorTok{,} \StringTok{" anos"}\OperatorTok{);}
\end{Highlighting}
\end{Shaded}

\begin{Shaded}
\begin{Highlighting}[]
\NormalTok{System}\OperatorTok{.}\AttributeTok{out}\OperatorTok{.}\FunctionTok{println}\NormalTok{(}\StringTok{"A idade mínima deve ser de "} \OperatorTok{+} \DecValTok{18} \OperatorTok{+} \StringTok{" anos"}\NormalTok{)}\OperatorTok{;}
\end{Highlighting}
\end{Shaded}

\begin{Shaded}
\begin{Highlighting}[]
\BuiltInTok{print}\NormalTok{(}\StringTok{"A idade mínima deve ser de "}\NormalTok{, }\DecValTok{18}\NormalTok{, }\StringTok{" anos"}\NormalTok{)}
\end{Highlighting}
\end{Shaded}

\begin{Shaded}
\begin{Highlighting}[]
\NormalTok{printf}\OperatorTok{(}\StringTok{"}\SpecialCharTok{\%s}\StringTok{ }\SpecialCharTok{\%d}\StringTok{ }\SpecialCharTok{\%s\textbackslash{}n}\StringTok{"}\OperatorTok{,} \StringTok{"A idade mínima deve ser de"}\OperatorTok{,} \DecValTok{18}\OperatorTok{,} \StringTok{"anos"}\OperatorTok{);}
\end{Highlighting}
\end{Shaded}

Saída na tela:

\begin{verbatim}
A idade mínima deve ser de 18 anos
\end{verbatim}

\subsection{Uso de armazenamento
intermediário}\label{uso-de-armazenamento-intermediuxe1rio}

No desenvolvimento de algoritmos, podemos utilizar armazenamento
intermediário de valores. Para tal, devemos indicar qual é o
\textbf{tipo de dado} que deverá ser guardado e um
\textbf{identificador} desse armazenamento.

:::info Informação O uso desse armazenamento é um tópico de grande
importância no desenvolvimento de algoritmos. O veremos de forma mais
detalhada ao abordarmos tipos primitivos de dados, constantes e
variáveis. :::

\textbf{Exemplo} - Armazenando um valor e apresentando na tela

\begin{Shaded}
\begin{Highlighting}[]
\NormalTok{inteiro}\OperatorTok{:}\NormalTok{ altura}\OperatorTok{;} \CommentTok{//identificador que permite armazenar um número inteiro}
\NormalTok{altura ← }\DecValTok{50}\OperatorTok{;}
\NormalTok{escreva}\OperatorTok{(}\NormalTok{altura}\OperatorTok{);} \CommentTok{//saída: 50}
\end{Highlighting}
\end{Shaded}

\begin{Shaded}
\begin{Highlighting}[]
\NormalTok{int altura}\OperatorTok{;}
\NormalTok{altura }\OperatorTok{=} \DecValTok{50}\OperatorTok{;} \CommentTok{//identificador que permite armazenar um número inteiro}
\NormalTok{System}\OperatorTok{.}\AttributeTok{out}\OperatorTok{.}\FunctionTok{println}\NormalTok{(altura)}\OperatorTok{;} \CommentTok{//saída: 50}
\end{Highlighting}
\end{Shaded}

\begin{Shaded}
\begin{Highlighting}[]
\NormalTok{altura }\OperatorTok{=} \DecValTok{50} \CommentTok{\#identificador que permite armazenar um número inteiro}
\BuiltInTok{print}\NormalTok{(altura) }\CommentTok{\# saída: 50}
\end{Highlighting}
\end{Shaded}

\begin{Shaded}
\begin{Highlighting}[]
\DataTypeTok{int}\NormalTok{ altura}\OperatorTok{;} \CommentTok{//identificador que permite armazenar um número inteiro}
\NormalTok{altura }\OperatorTok{=} \DecValTok{50}\OperatorTok{;}
\NormalTok{printf}\OperatorTok{(}\StringTok{"}\SpecialCharTok{\%d}\StringTok{"}\OperatorTok{,}\NormalTok{ altura}\OperatorTok{);} \CommentTok{// saída: 50}
\end{Highlighting}
\end{Shaded}

Saída na tela:

\begin{verbatim}
50
\end{verbatim}

É possível também apresentar diversos valores de uma única vez em uma
única saída.

\textbf{Exemplo}

\begin{Shaded}
\begin{Highlighting}[]
\NormalTok{caractere}\OperatorTok{:}\NormalTok{ nome ← }\StringTok{"Dunga"}\OperatorTok{;}
\NormalTok{inteiro}\OperatorTok{:}\NormalTok{ idade ← }\DecValTok{35}\OperatorTok{;}
\NormalTok{escreva}\OperatorTok{(}\NormalTok{nome}\OperatorTok{,} \StringTok{"tem"}\OperatorTok{,}\NormalTok{ idade}\OperatorTok{,} \StringTok{"anos de idade."}\OperatorTok{);}
\end{Highlighting}
\end{Shaded}

\begin{Shaded}
\begin{Highlighting}[]
\BuiltInTok{String}\NormalTok{ nome }\OperatorTok{=} \StringTok{"Dunga"}\OperatorTok{;}
\NormalTok{int idade }\OperatorTok{=} \DecValTok{35}\OperatorTok{;}
\NormalTok{System}\OperatorTok{.}\AttributeTok{out}\OperatorTok{.}\FunctionTok{println}\NormalTok{(nome }\OperatorTok{+} \StringTok{" tem "} \OperatorTok{+}\NormalTok{ anos }\StringTok{" de idade."}\NormalTok{)}\OperatorTok{;}
\end{Highlighting}
\end{Shaded}

\begin{Shaded}
\begin{Highlighting}[]
\NormalTok{nome }\OperatorTok{=}\NormalTok{ Dunga}
\NormalTok{idade }\OperatorTok{=} \DecValTok{35}
\BuiltInTok{print}\NormalTok{(}\SpecialStringTok{f"}\SpecialCharTok{\{}\NormalTok{nome}\SpecialCharTok{\}}\SpecialStringTok{ tem }\SpecialCharTok{\{}\NormalTok{idade}\SpecialCharTok{\}}\SpecialStringTok{ anos de idade."}\NormalTok{)}
\end{Highlighting}
\end{Shaded}

\begin{Shaded}
\begin{Highlighting}[]
\DataTypeTok{char}\NormalTok{ nome }\OperatorTok{=} \StringTok{"Dunga"}\OperatorTok{;}
\DataTypeTok{int}\NormalTok{ idade}\OperatorTok{;}

\NormalTok{idade }\OperatorTok{=} \DecValTok{35}\OperatorTok{;}
\NormalTok{printf}\OperatorTok{(}\StringTok{"}\SpecialCharTok{\%s}\StringTok{ }\SpecialCharTok{\%s}\StringTok{ }\SpecialCharTok{\%d}\StringTok{ }\SpecialCharTok{\%s}\StringTok{"}\OperatorTok{,}\NormalTok{nome}\OperatorTok{,}\StringTok{"tem"}\OperatorTok{,}\NormalTok{ idade}\OperatorTok{,} \StringTok{"anos de idade."}\OperatorTok{);}
\end{Highlighting}
\end{Shaded}

Saída na tela:

\begin{verbatim}
Dunga tem 35 anos de idade
\end{verbatim}

\textbf{Exercícios}

\begin{enumerate}
\def\labelenumi{\arabic{enumi}.}
\tightlist
\item
  Desenvolva um algoritmo que mostra na tela a mensagem \emph{Hoje é dia
  de aula de Algoritmos}.
\item
  Desenvolva um algoritmo que dentro dele possui um armazenamento
  intermediário de caracteres chamado \emph{mes\_ferias\_1} e
  \emph{mes\_ferias\_2}, onde cada um deve possuir os valores
  \emph{janeiro} e \emph{julho}. Seu algoritmo deve informar na tela uma
  mensagem informando que estes são os meses de férias previsto no
  calendário acadêmico.
\item
  Desenvolva um algoritmo que dentro dele possui os armazenamentos
  intermediários de caracteres chamado \emph{mes\_ferias\_1} e
  \emph{mes\_ferias\_2}, onde cada um deve possuir os valores
  \emph{janeiro} e \emph{julho}. Também possui os armazenamentos
  intermediários \emph{dias\_ferias\_1} e \emph{dias\_ferias\_2}, com os
  conteúdos \emph{30} e \emph{15}, respectivamente. Seu algoritmo deve
  informar na tela uma mensagem informando que são previstos 30 dias de
  férias em janeiro e 15 dias de férias em julho.
\end{enumerate}

\section{Entrada de dados}\label{entrada-de-dados}

Para que a entrada de dados possa ser realizado é necessário passar ao
algoritmo uma informação adicional, com o identificador em que o valor
deverá ser armazenado.

Para a entrada de dados pode-se utilizar a seguinte função:

\begin{Shaded}
\begin{Highlighting}[]
\NormalTok{leia}\OperatorTok{();} \CommentTok{//para qualquer tipo de dados}
\end{Highlighting}
\end{Shaded}

Antes de iniciar a leitura de dados, é necessário associar um
identificador à entrada.

\begin{Shaded}
\begin{Highlighting}[]
\NormalTok{Scanner entrada }\OperatorTok{=} \KeywordTok{new} \FunctionTok{Scanner}\NormalTok{(System}\OperatorTok{.}\AttributeTok{in}\NormalTok{)}\OperatorTok{;} \CommentTok{//entrada é um identificador}
\end{Highlighting}
\end{Shaded}

Em seguida é possível ler o dado desejado utilizando \texttt{.next()} ou
\texttt{.nextInt()}

\begin{Shaded}
\begin{Highlighting}[]
\NormalTok{entrada}\OperatorTok{.}\FunctionTok{next}\NormalTok{()}\OperatorTok{;} \CommentTok{//para dados do tipo caractere}
\end{Highlighting}
\end{Shaded}

ou

\begin{Shaded}
\begin{Highlighting}[]
\NormalTok{entrada}\OperatorTok{.}\FunctionTok{nextInt}\NormalTok{()}\OperatorTok{;} \CommentTok{//para dados do tipo inteiro}
\end{Highlighting}
\end{Shaded}

Finalizada a leitura de dados, deve-se chamar a função
\texttt{.close()}.

\begin{Shaded}
\begin{Highlighting}[]
\NormalTok{entrada}\OperatorTok{.}\FunctionTok{close}\NormalTok{()}\OperatorTok{;}
\end{Highlighting}
\end{Shaded}

:::caution Atenção Para que a função de entrada possa ser utilizada é
necessário importar a biblioteca também é necessário importar a
biblioteca \texttt{Scanner}.

Para tal, é necessário incluir no início do código a linha

\begin{Shaded}
\begin{Highlighting}[]
\ImportTok{import}\NormalTok{ java}\OperatorTok{.}\AttributeTok{util}\OperatorTok{.}\AttributeTok{Scanner}\OperatorTok{;}
\end{Highlighting}
\end{Shaded}

:::

:::caution Atenção O código
\texttt{Scanner\ entrada\ =\ new\ Scanner(System.in);} realiza uma
associação do identificador \texttt{entrada} com a entrada padrão
(comumente, o teclado).

Após a finalização das entradas, é necessário desassociar o
identificador, com a função \texttt{entrada.close().}

:::

\begin{Shaded}
\begin{Highlighting}[]
\BuiltInTok{input}\NormalTok{() }\CommentTok{\# para dados do tipo caractere}
\end{Highlighting}
\end{Shaded}

ou

\begin{Shaded}
\begin{Highlighting}[]
\NormalTok{(}\BuiltInTok{int}\NormalTok{) }\BuiltInTok{input}\NormalTok{() }\CommentTok{\# para dados do tipo inteiro}
\end{Highlighting}
\end{Shaded}

\begin{Shaded}
\begin{Highlighting}[]
\NormalTok{gets}\OperatorTok{();} \CommentTok{// entrada de dados do tipo caractere}
\end{Highlighting}
\end{Shaded}

ou
\texttt{c\ \ \ scanf("\%d",\ \&);\ //\ entrada\ de\ dados\ do\ tipo\ inteiro}

:::caution Atenção Assim como na função de saída \texttt{printf}, para
que as funções de entrada \texttt{scanf()} e \texttt{gets()} possam ser
utilizadas também é necessário importar a biblioteca de entrada e saída
padrão. :::

Conhecendo a função que realiza a leitura de dados da entrada padrão,
devemos informar qual identificador será responsável por armazenar o
dado recebido na entrada.

\textbf{Exemplo} - recebendo valores na entrada e armazenando

recebendo da entrada um valor do tipo inteiro

\begin{Shaded}
\begin{Highlighting}[]
\NormalTok{inteiro}\OperatorTok{:}\NormalTok{ numero}\OperatorTok{;}
\NormalTok{leia}\OperatorTok{(}\NormalTok{numero}\OperatorTok{);} 
\end{Highlighting}
\end{Shaded}

recebendo da entrada um valor do tipo caractere

\begin{Shaded}
\begin{Highlighting}[]
\NormalTok{caractere}\OperatorTok{:}\NormalTok{ palavra}\OperatorTok{;}
\NormalTok{leia}\OperatorTok{(}\NormalTok{palavra}\OperatorTok{);}
\end{Highlighting}
\end{Shaded}

recebendo da entrada um dado do tipo inteiro:

\begin{Shaded}
\begin{Highlighting}[]
\NormalTok{int numero}\OperatorTok{;}
\NormalTok{Scanner entrada }\OperatorTok{=} \KeywordTok{new} \FunctionTok{Scanner}\NormalTok{(System}\OperatorTok{.}\AttributeTok{in}\NormalTok{)}\OperatorTok{;}
\NormalTok{numero }\OperatorTok{=}\NormalTok{ entrada}\OperatorTok{.}\FunctionTok{nextInt}\NormalTok{()}\OperatorTok{;}
\NormalTok{entrada}\OperatorTok{.}\FunctionTok{close}\NormalTok{()}\OperatorTok{;}
\end{Highlighting}
\end{Shaded}

recebendo da entrada um dado do tipo caractere:
\texttt{javascript\ \ \ String\ palavra;\ \ \ Scanner\ entrada\ =\ new\ Scanner(System.in);\ \ \ palavra\ =\ entrada.next();\ \ \ entrada.close();}

recebendo da entrada um dado do tipo inteiro:
\texttt{python\ \ \ numero\ =\ (int)\ input()}

recebendo da entrada um dado do tipo caractere:
\texttt{python\ \ \ palavra\ =\ input()}

recebendo da entrada um dado do tipo inteiro
\texttt{c\ \ \ int\ numero;\ \ \ scanf("\%d",\ \&numero);\ //\ observe\ o\ \&}

recebendo da entrada um dado do tipo caractere
\texttt{c\ \ \ char\ palavra{[}100{]};\ \ \ gets(palavra);\ //perceba\ que\ o\ \&\ não\ é\ necessário\ aqui}

:::caution Atenção O uso da função \texttt{scanf()} requer cuidado ao
mencionar o identificador de onde o valor será armazenado. Observe o uso
do \texttt{\&} antes do nome do identificado. Para caracteres com a
função \texttt{gets()} este sinal não é necessário.

Estas diferenças e porque isto é realizado desta forma será explicado
futuramente, quando os conteúdos de vetores e manipulação de cadeias de
caracteres forem abordados. :::

\textbf{Exemplo}\\
- Solicite ao usuário que digite seu nome

```c showLineNumbers //identificadores caractere: nome;

\begin{verbatim}
//entrada
// highlight-next-line
leia(nome);   //recebe o dado da entrada
\end{verbatim}

\begin{verbatim}

<details>
<summary>Código completo</summary>

```c showLineNumbers
inicio;
  //identificadores
  caractere: nome;

  //entrada
  // highlight-next-line
  leia(nome);   //recebe o dado da entrada

fim.
\end{verbatim}

\begin{Shaded}
\begin{Highlighting}[]
\CommentTok{//identificadores}
\BuiltInTok{String}\NormalTok{ nome}\OperatorTok{;}

\CommentTok{//entrada}
\CommentTok{// highlight{-}next{-}line}
\NormalTok{nome }\OperatorTok{=}\NormalTok{ entrada}\OperatorTok{.}\FunctionTok{next}\NormalTok{()   }\CommentTok{//recebe o dado da entrada}
\end{Highlighting}
\end{Shaded}

Código completo

\begin{Shaded}
\begin{Highlighting}[]
  \KeywordTok{public} \KeywordTok{class}\NormalTok{ Main\{}
    \KeywordTok{public} \KeywordTok{static} \KeywordTok{void} \FunctionTok{main}\NormalTok{(}\BuiltInTok{String}\NormalTok{ args)\{}
      \CommentTok{//identificadores}
      \BuiltInTok{String}\NormalTok{ nome}\OperatorTok{;}
\NormalTok{      Scanner entrada}\OperatorTok{;}

      \CommentTok{//entrada}
\NormalTok{      entrada }\OperatorTok{=} \KeywordTok{new} \FunctionTok{Scanner}\NormalTok{(System}\OperatorTok{.}\AttributeTok{in}\NormalTok{)}\OperatorTok{;} \CommentTok{//associa o objeto à entrada padrão}

      \CommentTok{// highlight{-}next{-}line}
\NormalTok{      nome }\OperatorTok{=}\NormalTok{ entrada}\OperatorTok{.}\FunctionTok{next}\NormalTok{()   }\CommentTok{//recebe o dado da entrada}

\NormalTok{      entrada}\OperatorTok{.}\FunctionTok{close}\NormalTok{()         }\CommentTok{//finaliza a entrada de dados}
\NormalTok{    \}}
\NormalTok{  \}}
\end{Highlighting}
\end{Shaded}

A entrada de dados é a maneira como o usuário pode inserir dados para
dentro do algoritmo. Em Python utilizaremos a função \texttt{input()}.

\texttt{python\ showLineNumbers\ \ \ nome\ =\ input()}

Código completo

\begin{Shaded}
\begin{Highlighting}[]
    \ControlFlowTok{if} \VariableTok{\_\_name\_\_} \OperatorTok{==} \StringTok{"main"}\NormalTok{:}
\NormalTok{      nome }\OperatorTok{=} \BuiltInTok{input}\NormalTok{() }\CommentTok{\#recebe o dado da entrada}
\end{Highlighting}
\end{Shaded}

\begin{Shaded}
\begin{Highlighting}[]
\DataTypeTok{char}\NormalTok{ nome}\OperatorTok{[}\DecValTok{100}\OperatorTok{];}
\CommentTok{// highlight{-}next{-}line}
\NormalTok{gets}\OperatorTok{(}\NormalTok{nome}\OperatorTok{);} \CommentTok{//recebe o dado da entrada}
\end{Highlighting}
\end{Shaded}

Código completo

\begin{Shaded}
\begin{Highlighting}[]
    \PreprocessorTok{\#include }\ImportTok{\textless{}stdio.h\textgreater{}}

    \DataTypeTok{int}\NormalTok{ main}\OperatorTok{()\{}
      \DataTypeTok{char}\NormalTok{ nome}\OperatorTok{[}\DecValTok{100}\OperatorTok{];}
    \CommentTok{// highlight{-}next{-}line}
\NormalTok{      gets}\OperatorTok{(}\NormalTok{nome}\OperatorTok{);} \CommentTok{//recebe o dado da entrada}

      \ControlFlowTok{return} \DecValTok{0}\OperatorTok{;}
    \OperatorTok{\}}
\NormalTok{    \textasciigrave{}\textasciigrave{}\textasciigrave{}}
  \OperatorTok{\textless{}/}\NormalTok{details}\OperatorTok{\textgreater{}}

  \OperatorTok{\textless{}/}\NormalTok{TabItem}\OperatorTok{\textgreater{}}

\OperatorTok{\textless{}/}\NormalTok{Tabs}\OperatorTok{\textgreater{}}


\OperatorTok{**}\NormalTok{Observação}\OperatorTok{**}  

\OperatorTok{{-}}\NormalTok{ Perceba que no exemplo}\OperatorTok{,}\NormalTok{ o nome }\ControlFlowTok{do}\NormalTok{ usuário será armazenado na região de armazenamento identificada como \textasciigrave{}nome\textasciigrave{}}\OperatorTok{.}

\OperatorTok{**}\NormalTok{Exemplo}\OperatorTok{**}
\OperatorTok{{-}}\NormalTok{ Pergunte ao usuário seu nome e idade}\OperatorTok{.}\NormalTok{ Em seguida}\OperatorTok{,}\NormalTok{ mostre na tela as informações digitadas}\OperatorTok{.}

\OperatorTok{\textless{}}\NormalTok{Tabs groupId}\OperatorTok{=}\StringTok{"language"}\OperatorTok{\textgreater{}}
  \OperatorTok{\textless{}}\NormalTok{TabItem value}\OperatorTok{=}\StringTok{"pseudocodigo"}\NormalTok{ label}\OperatorTok{=}\StringTok{"Pseudocódigo"} \ControlFlowTok{default}\OperatorTok{\textgreater{}}

\NormalTok{  \textasciigrave{}\textasciigrave{}\textasciigrave{}c}
  \CommentTok{//Identificadores}
\NormalTok{  caractere nome}\OperatorTok{;}
\NormalTok{  inteiro idade}\OperatorTok{;}

  \CommentTok{//Entrada}
\NormalTok{  escreva}\OperatorTok{(}\StringTok{"Nome: "}\OperatorTok{);}
\NormalTok{  leia}\OperatorTok{(}\NormalTok{nome}\OperatorTok{);}
\NormalTok{  escreva}\OperatorTok{(}\StringTok{"Idade: "}\OperatorTok{);}
\NormalTok{  leia}\OperatorTok{(}\NormalTok{idade}\OperatorTok{);}

  \CommentTok{//Saída}
\NormalTok{  escreva}\OperatorTok{(}\NormalTok{nome}\OperatorTok{,} \StringTok{" tem "}\OperatorTok{,}\NormalTok{ idade}\OperatorTok{,} \StringTok{" anos de idade."}\OperatorTok{);}
\end{Highlighting}
\end{Shaded}

Código completo

\begin{Shaded}
\begin{Highlighting}[]
\NormalTok{  inicio}

    \CommentTok{//Variáveis}
\NormalTok{    caractere nome}\OperatorTok{;}
\NormalTok{    inteiro idade}\OperatorTok{;}

    \CommentTok{//Entrada}
\NormalTok{    escreva}\OperatorTok{(}\StringTok{"Nome: "}\OperatorTok{);}
\NormalTok{    leia}\OperatorTok{(}\NormalTok{nome}\OperatorTok{);}
\NormalTok{    escreva}\OperatorTok{(}\StringTok{"Idade: "}\OperatorTok{);}
\NormalTok{    leia}\OperatorTok{(}\NormalTok{idade}\OperatorTok{);}

    \CommentTok{//Saída}
\NormalTok{    escreva}\OperatorTok{(}\NormalTok{nome}\OperatorTok{,} \StringTok{" tem "}\OperatorTok{,}\NormalTok{ idade}\OperatorTok{,} \StringTok{" anos de idade."}\OperatorTok{);}

\NormalTok{  fim}\OperatorTok{.}
\end{Highlighting}
\end{Shaded}

\begin{Shaded}
\begin{Highlighting}[]
  \CommentTok{//Identificadores}
  \BuiltInTok{String}\NormalTok{ nome}\OperatorTok{;}
\NormalTok{  int idade}\OperatorTok{;}

  \CommentTok{//Entrada}
\NormalTok{  System}\OperatorTok{.}\AttributeTok{out}\OperatorTok{.}\FunctionTok{println}\NormalTok{(}\StringTok{"Nome: "}\NormalTok{)}\OperatorTok{;}
\NormalTok{  nome }\OperatorTok{=}\NormalTok{ entrada}\OperatorTok{.}\FunctionTok{next}\NormalTok{()}\OperatorTok{;}
\NormalTok{  System}\OperatorTok{.}\AttributeTok{out}\OperatorTok{.}\FunctionTok{println}\NormalTok{(}\StringTok{"Idade: "}\NormalTok{)}\OperatorTok{;}
\NormalTok{  idade }\OperatorTok{=}\NormalTok{ entrada}\OperatorTok{.}\FunctionTok{nextInt}\NormalTok{()}\OperatorTok{;}

  \CommentTok{//Saída}
\NormalTok{  System}\OperatorTok{.}\AttributeTok{out}\OperatorTok{.}\FunctionTok{prinln}\NormalTok{(nome }\OperatorTok{+} \StringTok{" tem "} \OperatorTok{+}\NormalTok{ idade }\OperatorTok{+} \StringTok{" anos de idade."}\NormalTok{)}\OperatorTok{;}
\end{Highlighting}
\end{Shaded}

Código completo

\begin{Shaded}
\begin{Highlighting}[]
  \KeywordTok{public} \KeywordTok{class}\NormalTok{ Main\{}
    \KeywordTok{public} \KeywordTok{static} \KeywordTok{void} \FunctionTok{main}\NormalTok{(}\BuiltInTok{String}\NormalTok{ args)\{}
      \CommentTok{//Variáveis}
      \BuiltInTok{String}\NormalTok{ nome}\OperatorTok{;}
\NormalTok{      int idade}\OperatorTok{;}

      \CommentTok{//Entrada}
\NormalTok{      Scanner entrada }\OperatorTok{=} \KeywordTok{new} \FunctionTok{Scanner}\NormalTok{(System}\OperatorTok{.}\AttributeTok{in}\NormalTok{)}\OperatorTok{;}

\NormalTok{      System}\OperatorTok{.}\AttributeTok{out}\OperatorTok{.}\FunctionTok{println}\NormalTok{(}\StringTok{"Nome: "}\NormalTok{)}\OperatorTok{;}
\NormalTok{      nome }\OperatorTok{=}\NormalTok{ entrada}\OperatorTok{.}\FunctionTok{next}\NormalTok{()}\OperatorTok{;}
\NormalTok{      System}\OperatorTok{.}\AttributeTok{out}\OperatorTok{.}\FunctionTok{println}\NormalTok{(}\StringTok{"Idade: "}\NormalTok{)}\OperatorTok{;}
\NormalTok{      idade }\OperatorTok{=}\NormalTok{ entrada}\OperatorTok{.}\FunctionTok{nextInt}\NormalTok{()}\OperatorTok{;}
\NormalTok{      entrada}\OperatorTok{.}\FunctionTok{close}\NormalTok{()}

      \CommentTok{//Saída}
\NormalTok{      System}\OperatorTok{.}\AttributeTok{out}\OperatorTok{.}\FunctionTok{prinln}\NormalTok{(nome }\OperatorTok{+} \StringTok{" tem "} \OperatorTok{+}\NormalTok{ idade }\OperatorTok{+} \StringTok{" anos de idade."}\NormalTok{)}\OperatorTok{;}
\NormalTok{    \}}
\NormalTok{  \}}
\end{Highlighting}
\end{Shaded}

\begin{Shaded}
\begin{Highlighting}[]
  \CommentTok{\#Entrada}
\NormalTok{  nome }\OperatorTok{=} \BuiltInTok{input}\NormalTok{(}\StringTok{"Nome: "}\NormalTok{)}
\NormalTok{  idade }\OperatorTok{=} \BuiltInTok{input}\NormalTok{(}\StringTok{"Idade: "}\NormalTok{)}

  \CommentTok{\#Saída}
  \BuiltInTok{print}\NormalTok{(}\SpecialStringTok{f"}\SpecialCharTok{\{}\NormalTok{nome}\SpecialCharTok{\}}\SpecialStringTok{ tem }\SpecialCharTok{\{}\NormalTok{idade}\SpecialCharTok{\}}\SpecialStringTok{ anos de idade"}\NormalTok{)}
\end{Highlighting}
\end{Shaded}

Código completo

\begin{Shaded}
\begin{Highlighting}[]
    \ControlFlowTok{if} \VariableTok{\_\_name\_\_} \OperatorTok{==} \StringTok{"main"}\NormalTok{:}
      \CommentTok{\#Entrada}
\NormalTok{      nome }\OperatorTok{=} \BuiltInTok{input}\NormalTok{(}\StringTok{"Nome: "}\NormalTok{)}
\NormalTok{      idade }\OperatorTok{=} \BuiltInTok{input}\NormalTok{(}\StringTok{"Idade: "}\NormalTok{)}

      \CommentTok{\#Saída}
      \BuiltInTok{print}\NormalTok{(}\SpecialStringTok{f"}\SpecialCharTok{\{}\NormalTok{nome}\SpecialCharTok{\}}\SpecialStringTok{ tem }\SpecialCharTok{\{}\NormalTok{idade}\SpecialCharTok{\}}\SpecialStringTok{ anos de idade"}\NormalTok{)}
\NormalTok{    \textasciigrave{}\textasciigrave{}\textasciigrave{}}
  \OperatorTok{\textless{}/}\NormalTok{details}\OperatorTok{\textgreater{}}


  \OperatorTok{\textless{}/}\NormalTok{TabItem}\OperatorTok{\textgreater{}}

  \OperatorTok{\textless{}}\NormalTok{TabItem value}\OperatorTok{=}\StringTok{"c"}\NormalTok{ label}\OperatorTok{=}\StringTok{"C"}\OperatorTok{\textgreater{}}

\NormalTok{  \textasciigrave{}\textasciigrave{}\textasciigrave{}c}
  \OperatorTok{//}\NormalTok{Variáveis}
\NormalTok{  char nome[}\DecValTok{100}\NormalTok{]}\OperatorTok{;}
  \BuiltInTok{int}\NormalTok{ idade}\OperatorTok{;}

  \OperatorTok{//}\NormalTok{Entrada}
\NormalTok{  gets(nome)}\OperatorTok{;}
\NormalTok{  scanf(}\StringTok{"}\SpecialCharTok{\%d}\StringTok{"}\NormalTok{, }\OperatorTok{\&}\NormalTok{idade)}

  \OperatorTok{//}\NormalTok{Saída}
\NormalTok{  printf(}\StringTok{"}\SpecialCharTok{\%s}\StringTok{ }\SpecialCharTok{\%s}\StringTok{ }\SpecialCharTok{\%d}\StringTok{ }\SpecialCharTok{\%s}\CharTok{\textbackslash{}n}\StringTok{"}\NormalTok{, nome, }\StringTok{"tem"}\NormalTok{, idade, }\StringTok{"anos de idade"}\NormalTok{)}
\end{Highlighting}
\end{Shaded}

Código completo

\begin{Shaded}
\begin{Highlighting}[]
  \PreprocessorTok{\#include}\ImportTok{\textless{}stdio.h\textgreater{}}

  \DataTypeTok{int}\NormalTok{ main}\OperatorTok{(}\DataTypeTok{void}\OperatorTok{)\{}
    \CommentTok{//Variáveis}
    \DataTypeTok{char}\NormalTok{ nome}\OperatorTok{[}\DecValTok{100}\OperatorTok{];}
    \DataTypeTok{int}\NormalTok{ idade}\OperatorTok{;}

    \CommentTok{//Entrada}
\NormalTok{    gets}\OperatorTok{(}\NormalTok{nome}\OperatorTok{);}
\NormalTok{    scanf}\OperatorTok{(}\StringTok{"}\SpecialCharTok{\%d}\StringTok{"}\OperatorTok{,} \OperatorTok{\&}\NormalTok{idade}\OperatorTok{)}

    \CommentTok{//Saída}
\NormalTok{    printf}\OperatorTok{(}\StringTok{"}\SpecialCharTok{\%s}\StringTok{ }\SpecialCharTok{\%s}\StringTok{ }\SpecialCharTok{\%d}\StringTok{ }\SpecialCharTok{\%s\textbackslash{}n}\StringTok{"}\OperatorTok{,}\NormalTok{ nome}\OperatorTok{,} \StringTok{"tem"}\OperatorTok{,}\NormalTok{ idade}\OperatorTok{,} \StringTok{"anos de idade"}\OperatorTok{)}
  \OperatorTok{\}}
\end{Highlighting}
\end{Shaded}

\textbf{Observação} - No exemplo, o nome do usuário será armazenado no
espaço identificado como \texttt{nome} e a idade em \texttt{idade}.

\textbf{Exercícios}

\begin{enumerate}
\def\labelenumi{\arabic{enumi}.}
\item
  Solicite ao usuário que digite um número. Em seguida, mostre na tela o
  número digitado.
\item
  Desenvolva um algoritmo que pergunta ao usuário o nome do usuário, e
  em seguida responde ``Boa noite, \texttt{user}'', substituindo
  \texttt{user} pelo nome digitado.
\item
  Faça um algoritmo que pergunta ao usuário a sua idade, e em seguida
  informa a mensagem ``Você tem \texttt{X} anos'', substituindo
  \texttt{X} pela idade digitada.
\end{enumerate}

\bookmarksetup{startatroot}

\chapter{Sumário}\label{sumuxe1rio}

In summary, this book has no content whatsoever.

\bookmarksetup{startatroot}

\chapter*{References}\label{references}
\addcontentsline{toc}{chapter}{References}

\markboth{References}{References}

\phantomsection\label{refs}
\begin{CSLReferences}{0}{1}
\end{CSLReferences}



\end{document}
